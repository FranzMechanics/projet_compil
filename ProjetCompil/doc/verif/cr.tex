\part*{CS444 - Projet Compilation \\ Documentation}


\section*{Les messages d'erreurs}

    Pour la gestion des erreurs nous avons à disposition ErreurContext.java,
     ErreurInterneVerif.java et ErreurReglesTypage.java.

     \vspace{3mm}

     Pour la suite nous avons préferé gerer les erreurs avec la classe enum ErreurContext et non par la classe ErreurReglesTypage.
En effet, cela nous permet de différencier les erreurs levées et d'afficher des messages spécifiques.
D'autre part, si la passe 1 a été correctement faite, il ne devrait pas avoir d'ErreurInterneVerif.

    \vspace{3mm}

    Voici les erreurs contextuelles ainsi que leur description et le message affiché :

    \vspace{3mm}

    - ErreurVariableRedeclaree:

    Description: Lors de la déclaration des variables, si une variable a le meme nom qu'une variable déjà déclarée

    Message: Erreur contextuelle : Variable non déclarée

    Variable [nom\_variable] non déclarée ... ligne [num\_ligne]

    \vspace{3mm}

    - ErreurVariableNonDeclaree

    Description: Si on utilise une variable qui n'a pas été déclarée au préalable

    Message: Erreur contextuelle : Variable non déclarée

    Variable [nom\_variable] non déclarée ... ligne [num\_ligne]

    \vspace{3mm}

    - ErreurTypageNonCompatible

    Description: Lors d'une affectation, une operation binaire ou une operation unaire et que le type attendu n'est pas le bon.

    Message: Erreur contextuelle : Opération non compatible :

    Opération Affect : Types [type1] et [type2] incompatibles ... ligne [num\_ligne]

    \vspace{3mm}

    - ErreurTypeIndefini

    Description: Lors d'une déclaration d'une variable, on lui donne un type indefini

    Message: Erreur contextuelle : Typeindefini

    Le type[type\_indefini] est indefini dans l'environnement ... ligne [num\_ligne]

    \vspace{3mm}

    - ErreurIdentNomReserve

    Description: Lors d'une utilisation d'un mot reservé

    Message: Erreur contextuelle : Identificateur déclaré avec un nom réservé

    [mot\_reservé] est un nom réservé ... ligne [num\_ligne]

    \vspace{3mm}

    - ErreurNonRepertoriee

    Description: Aucune des erreurs précédentes, normalement on utilise jamais cette erreur.

    Message: Erreur contextuelle : Erreur non repertoriee

    \vspace{3mm}


\section*{La méthodologie de test}

    L'objectif principale de cette base de test est de tester notre compilateur
    sur des régles sémantiques décrites dans le fichier context.txt.

    \vspace{3mm}

    Pour ce faire, nous avons respecter certaines conditions afin de couvrir
    la majeur partie de règles.

    \vspace{3mm}

    \begin{itemize}
        \item Bien décomposer les problèmes en écrivant des méthodes COURTES.
        \item Factoriser les éléments communs (Eviter les tests testant les mêmes régles sémantiques)
        \item Documenter chaque test à l'aide d'un commentaire avant le code testé
    \end{itemize}

    \vspace{3mm}

    On écrira des programmes JCas valides et invalides sémantiquement.

    \vspace{3mm}

       --> Dans le cas valide:

    \tab   Le programme s'éxecute correctement, on vérifie que l'arbre
            est correctement décoré.

    \vspace{3mm}

        --> Dans le cas invalide:

    \tab     - On vérifie que le message d'erreur est pertinent.

    \vspace{10mm}

    Une fois le fichier \emph{context.txt} assimilé, nous avons réalisé une liste
    de tests exhaustive suivant chacune des régles définies.

    \vspace{3mm}

    Pour chaques régles, nous avons écrits des tests valides sémantiquement, et des
    tests invalides afin de couvrir le plus de cas possibles.

    \vspace{3mm}
