\part*{CS444 - Projet Compilation \\ Documentation}


\section*{Base de test}

    L'objectif principale de cette base de test est de tester notre compilateur
    sur des régles sémantiques décrites dans le fichier context.txt.

    \vspace{3mm}

    Pour ce faire, nous avons respecter certaines conditions afin de couvrir
    la majeur partie de règles.

    \vspace{3mm}

    \begin{itemize}
        \item Bien décomposer les problèmes en écrivant des méthodes COURTES.
        \item Factoriser les éléments communs (Eviter les tests testant les mêmes régles sémantiques)
        \item Documenter chaque test à l'aide d'un commentaire avant le code testé
    \end{itemize}

    \vspace{3mm}

    On écrira des programmes JCas valides et invalides sémantiquement.

    \vspace{3mm}

       --> Dans le cas valide:

    \tab   Le programme s'éxecute correctement, on vérifie que l'arbre
            est correctement décoré.

    \vspace{3mm}

        --> Dans le cas invalide:

    \tab     - On vérifie que le message d'erreur est pertinent.

    \vspace{10mm}

    Une fois le fichier \emph{context.txt} assimilé, nous avons réalisé une liste
    de tests exhaustive suivant chacune des régles définies.

    \vspace{3mm}

    Pour chaques régles, nous avons écrits des tests valides sémantiquement, et des
    tests invalides afin de couvrir le plus de cas possibles.

    \vspace{3mm}

    
